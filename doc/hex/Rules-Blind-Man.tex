\documentclass[a4paper,12pt]{article}
\usepackage{graphicx,amsfonts,url}
\usepackage[utf8]{inputenc}
\title{Rules of the Blind man's buff game}
\author{}
\date{2013}

\setlength{\parindent}{0pt}
\setlength{\parskip}{2ex}

\begin{document}
\maketitle

\section*{Overview}

The game is played in an environment where there is a hexagonal grid of $n$~x~$n$ cells, each potentially containing
objects. The boundary cells contain wall, some cells contain obstacle, others can contain person. The game is
played by one agent whose task is to localize any other person, if possible. At the start, the agent has in
\texttt{(bs-agent6-body-grid agent-body)} the full map of walls, persons and obstacles. The
dimensions of the grid can be discovered using the standard function \texttt{array-dimensions}. The agent does not know its position, neither the direction it is facing at the start.

\section*{Permissible actions of an agent}

\textbf{Moves} are processed provided that the target cell is empty:

\begin{itemize}
\item \texttt{north} -- move one cell to the north in the direction which the agent is facing.
\item \texttt{northwest} -- move one cell to the  north-west of the direction which the agent is facing.
\item \texttt{northeast} -- move one cell to the north-east of the direction which the agent is facing.
\item \texttt{south} -- move one cell to the south of the direction which the agent is facing.
\item \texttt{southwest} -- move one cell to the south-west of the direction which the agent is facing.
\item \texttt{southeast} -- move one cell to the south-east of the direction which the agent is facing.
\end{itemize}

\textbf{Stopping the game}
\begin{itemize}
\item \texttt{stop} -- if  no person can be localized.
\end{itemize}

\section*{Structure of agent's percept}
The agent's percept is a two-element list, whose first element is the \texttt{agent-body}. The second element is a
proper agent's percept of the external world. At the start, this percept is \texttt{nil}, in every following step the
percept reflects the previous action as follows:
\begin{itemize}
\item \texttt{nil} -- if the previous action was successfully performed.
\item \texttt{bump} -- if the previous action was not successfully performed (the target space contains wall or
obstacle).
\end{itemize}

If the agent moves to a cell containing a person, the game is stopped. The task of the agent is to find a person as
fast as possible or to find out that it is not possible (due to obstacles). The score of the agent is calculated as a
negative number of steps needed to finish the game. If the agent stops the game incorrectly, the score is $-1000$.

\section*{How to run the game}
\begin{enumerate}
\item The directory in the second line of the file \texttt{run.lisp} must be changed to match the project root directory, e.g.
\begin{verbatim}
(cd "~/agentsim")
\end{verbatim}

\item Agent test runs can be performed using

\begin{verbatim}
(run-gui (make-bs-world6))
\end{verbatim}

\end{enumerate}

\section*{\texttt{ask-user-agent} key bindings}
You can use keys $Q$, $W$ and $E$ to move north-west, north and north-east. Keys $A$, $S$ and $D$ move the agent to the
south-west, south and south-east. Pressing $X$ stops the game.

\section*{List of useful functions}
Following functions can be found in file \texttt{hex/utilities.lisp} and should be examined before usage:
\begin{itemize}
\item \texttt{xy-update}
\item \texttt{tnorth}
\item \texttt{tnorthwest}
\item \texttt{tnortheast}
\item \texttt{tsouth}
\item \texttt{tsouthwest}
\item \texttt{tsoutheast}
\end{itemize}

\end{document}
