\documentclass[a4paper,12pt]{article}
\usepackage{graphicx,amsfonts,url}
\usepackage[utf8]{inputenc}
\title{Rules of the Hide\&Seek game}
\author{}
\date{2013}

\setlength{\parindent}{0pt}
\setlength{\parskip}{2ex}

\begin{document}
\maketitle

\section*{Overview}

The game is played in an environment where there is a hexagonal grid of $n$ x $n$ cells. All boundary cells
contain a wall. Some internal cells contain bush and in some bushes there is a hidden person; other cells can contain an
obstacle. The game is played by one agent whose task is to find as fast as possible all hidden persons (or all hidden
persons that can be found). Initial position of  the agent is \texttt{(1 1)}, locations of the remaining objects in the
grid are chosen randomly.

\section*{Permissible actions of an agent}

\textbf{Moves} are processed provided that the target cell is empty:

\begin{itemize}
\item \texttt{north} -- move one cell to the north in the direction which the agent is facing.
\item \texttt{northwest} -- move one cell to the  north-west of the direction which the agent is facing.
\item \texttt{northeast} -- move one cell to the north-east of the direction which the agent is facing.
\end{itemize}

\textbf{Possible complementary move actions}
\begin{itemize}
\item All move actions can be specified as $n$-times repeated in the form \texttt{(action n)}, where $n$ is a positive
integer. This action corresponds to the basic action being repeated exactly $n$-times at one step. It will be stopped
when an obstacle is reached.
\end{itemize}

\textbf{Turns} are actions that change the direction which the agent is facing:
\begin{itemize}
\item \texttt{turnleft} -- turn by $60^\circ$ to the left.
\item \texttt{turnright} -- turn by $60^\circ$ to the right.
\end{itemize}

\textbf{Finding}
\begin{itemize}
\item \texttt{pyky} -- finding a person hidden in a bush is possible only provided that:
\begin{itemize}
\item The bush can be seen by the agent.
\item The bush is facing towards the agent.
\item The agent is facing towards the bush.
\end{itemize}
\end{itemize}

\textbf{Stopping the game}
\begin{itemize}
\item \texttt{stop} -- if no person can be localized.
\end{itemize}

\section*{Structure of agent's percept}
The agent's percept is a variable-length list, whose first element is the agent-body. The second element is a list
containing one element, which is the contents of the cell one step to the north of the direction, in which the agent is
facing. The $(n+1)$-th element is a list containing $n$ elements, which correspond to the positions visible from
the agent inside the angle $120^\circ$ in the direction where the agent is facing. Content of the cell is represented as follows:

\begin{itemize}
\item \texttt{nil} -- the cell is empty.
\item \texttt{dark} -- the cell cannot be seen by the agent, because it is blocked out by an obstacle (wall or bush).
\item \texttt{person} -- agent is looking at a bush with a hidden person. The bush is facing towards the agent.
\item \texttt{bush} -- agent is looking towards a bush in which nobody is hidden or which is facing in some other direction.
\item \texttt{wall} -- agent is looking at a wall.
\end{itemize}

Visibility of cells in percept is determined by the Breadth-first search algorithm run from the bottom and left to
right. Cell added to the algorithm queue by a visible cell will also be visible, while the cell enqueued by a
\texttt{dark} cell will be \texttt{dark}. Try the game to see how it works.

\section*{State of the world}
The agent creates its representation of the state of the external world in the field \texttt{(hs-agent6-body-grid
agent-body)}.

\section*{How to run the game}
\begin{enumerate}
\item The directory in the second line of the file \texttt{run.lisp} must be changed to match the project root directory, e.g.
\begin{verbatim}
(cd "~/agentsim")
\end{verbatim}

\item Agent test runs can then be performed using

\begin{verbatim}
(run-gui (make-hs-world6))
\end{verbatim}

\end{enumerate}

\section*{\texttt{ask-user-agent} key bindings}
You can use keys $Q$, $W$ and $E$ to move north-west, north and north-east. Keys $A$ and $D$ turn
the agent to the left and to the right. Pressing $P$ causes the agent to \texttt{pyky}, $X$ stops the game.

\section*{List of useful functions}
Following functions can be found in file \texttt{hex/utilities.lisp} and should be examined before usage:
\begin{itemize}
\item \texttt{count-persons}
\item \texttt{update-grid}
\item \texttt{xy-update}
\item \texttt{tnorth}
\item \texttt{tnorthwest}
\item \texttt{tnortheast}
\item \texttt{tsouth}
\item \texttt{tsouthwest}
\item \texttt{tsoutheast}
\end{itemize}

\end{document}
